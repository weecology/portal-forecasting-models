\documentclass{article}

\usepackage[english]{babel}
\usepackage[utf8]{inputenc}
\usepackage{lmodern}
\usepackage[autostyle]{csquotes}
\usepackage{amsmath}
\usepackage{graphicx}
\usepackage[colorinlistoftodos]{todonotes}
\usepackage[T1]{fontenc}
\usepackage[utf8]{inputenc}
\usepackage{authblk}
\usepackage{hyperref}
\usepackage{indentfirst}
\usepackage{bbold}
\def\code#1{\texttt{#1}}
\hypersetup{
    colorlinks = true,
    linkcolor = black,
    filecolor = black,      
    urlcolor = black,
    citecolor = black,
}
\usepackage{natbib}
\bibliographystyle{abbrvnat}
\setcitestyle{authoryear,open={(},close={)}}

\title{Portal Forecasting Models}


\author[1]{Weecology Laboratory}
\affil[1]{University of Florida}

\date{\today}

\begin{document}
\maketitle
\tableofcontents

\addcontentsline{toc}{section}{Overview}
\section*{Overview}
\label{sec:overview}

This document describes the models used to forecast ecological predictions for the Portal Project \citep{Brown1998, PortalPredictions}. 

\section{Current Models}
\label{sec:currmods}

\subsection{Na\"{i}ve 1}
\label{subsec:naive01}

"Na\"{i}ve 1" is a flexible exponential smoothing state space (ESSS) model \citep{Hyndman2008} fit to the data at the composite (site) spatial level and the composite (community) ecological level. The model is selected and fitted using the \code{ets} and \code{forecast} functions in the \textbf{forecast} package \citep{Hyndman2017} with the \code{allow.multiplicative.trend} argument set to \code{TRUE} and the \code{naive01} function in our \textbf{portalPredictions} package \citep{PortalPredictions}. ESSS models fit using \code{ets} implement what is known as the "innovations" approach to state space modeling, which assumes a single source of noise that is equivalent for the process and observation errors \citep{Hyndman2008}.

ESSS models are defined according to three model structure parameters: error type, trend type, and seasonality type \citep{Hyndman2008}. Each of the parameters can be an N (none), A (additive), or M (multiplicative) state \citep{Hyndman2008}. "Na\"{i}ve 1" is fit flexibly, such that the model parameters can vary from fit to fit.

\subsection{Na\"{i}ve 2}
\label{subsec:naive02}

"Na\"{i}ve 2" is a flexible Auto-Regressive Integrated Moving Average (ARIMA) model fit to the data at the composite (site) spatial level and the composite (community) ecological level. The model is selected and fitted using the \code{auto.arima} (with the \code{lambda} argument set to 0) and \code{forecast} functions in the \textbf{forecast} package \citep{Hyndman2013, Hyndman2017} and the \code{naive02} function in our \textbf{portalPredictions} package \citep{PortalPredictions}.

ARIMA models are defined according to three model structure parameters: the number of autoregressive terms (p), the degree of differencing (d), and the order of the moving average (q), and are represented as ARIMA(p, q, q) \citep{Box1970}. "Na\"{i}ve 2" is fit flexibly, such that the model parameters can vary from fit to fit.

\subsection{Negative Binomial TS}
\label{subsec:neg_binom_ts}

"Negative Binomial TS" (NBTS) is a generalized autoregressive conditional heteroskedasticity (GARCH) model with overdispersion fit to the data at the composite (site) spatial level and the articulated (species) ecological level. The model for each species is selected and fitted using the \code{tsglm} function in the \textbf{tscount} package \citep{Liboschik2017a} and the \code{neg\_binom\_ts} function in our \textbf{portalPredictions} package \citep{PortalPredictions}.

GARCH models are generalized ARMA models and are defined according to their link function, response distribution, and two model structure parameters: the number of autoregressive terms (p) and the order of the moving average (q), and are represented as GARCH(p, q) \citep{Liboschik2017a}. The NBTS model is fit using the log link and a negative binomial response (modeled as an over-dispersed Poisson), as well as with p = 1 (first-order autoregression) and q = 12 (yearly moving average).

The \code{tsglm} function in the \textbf{tscount} package \citep{Liboschik2017a} uses a (conditional) quasi-likelihood based approach to inference and models the overdispersion as an additional parameter in a two-step approach. This two-stage approach has only been minimally evaluated, although preliminary simulation-based studies are promising \citep{Liboschik2017b}.    

Species are fit individually and then summed to produce a composite estimate. 

\subsection{Poisson Environmental TS}
\label{subsec:pois_env_ts}

"Poisson Environmental TS" (PETS) is a generalized autoregressive conditional heteroskedasticity (GARCH) model fit to the data at the composite (site) spatial level and the articulated (species) ecological level. The model for each species is selected and fitted using the \code{tsglm} function in the \textbf{tscount} package \citep{Liboschik2017a} and the \code{pois\_env\_ts} function in our \textbf{portalPredictions} package \citep{PortalPredictions}.

GARCH models are generalized ARMA models and are defined according to their link function, response distribution, and two model structure parameters: the number of autoregressive terms (p) and the order of the moving average (q), and are represented as GARCH(p, q) \citep{Liboschik2017a}. The PETS model is fit using the log link and a Poisson response, as well as with p = 1 (first-order autoregression) and q = 12 (yearly moving average). The environmental variables potentially includes in the model are min, mean, and max temperatures, precipiation, and NDVI. 

The \code{tsglm} function in the \textbf{tscount} package \citep{Liboschik2017a} uses a (conditional) quasi-likelihood based approach to inference. This approach has only been minimally evaluated for models with covariates, although preliminary simulation-based studies are promising \citep{Liboschik2017b}.  

Each species is fit using the following (nonexhaustive) sets of the environmental covariates:
\begin{itemize}
\item max temp, mean temp, precipitation, NDVI
\item max temp, min temp, precipitation, NDVI
\item max temp, mean temp, min temp, precipitation
\item precipitation, NDVI
\item min temp, NDVI
\item min temp
\item max temp
\item mean temp
\item precipitation 
\item NDVI
\end{itemize}

Currently, there is no intercept-only model included. The single best model of the 10 is selected based on AIC. Species are fit individually and then summed to produce a composite estimate. 

\section{Developing Models}
\label{sec:devmods}

\subsection{General Framework}
\label{subsec:devmods_gf}

The framework I propose here is based on a data-assimilation approach to forecasting using dynamical state-space models (a.k.a. partially observed Markov process models), whereby the observed data are considered a partial sample of a "true" state that evolves through time \citep{Ionides2006, King2016}, and the models used to estimate and forecast the true state are iteratively evaluated and updated with each new observation. We use a suite of competing models to make inference about the historic and current state of the system and to produce forecasts of expected future observations. Ongoing data collections provide new observations, which are used to iteratively evaluate the models and produce new predictions via the forecast cycle \citep{Dietze2017}. 

We use $y$ to represent an observation and $y_{ijk}$ is the observed number of rodents of species $i$ in plot $j$ at survey $k$. Because $y_{ijk}$ is a count of individuals, the variable is non-negative and integer-valued. The observed numbers of individuals across each species $i$ in $1 \ldots I$ and each plot $j$ in $1 \ldots J$ for a given survey $k$ is an $I \times J$ matrix $\textbf{Y}_k$. The series of observations for surveys $1 \ldots K$ is given by $\textbf{Y}_{1:K}$. We use $\textbf{Y}_{1:K}$ to evaluate all possible models ($m$ in $1 \ldots M$ total models). The total likelihood ($\mathcal{L}(\mathcal{M}_m|\textbf{Y}_{1:K})$) of a given model $m$ is the data-based measure of support for that model $\mathcal{M}_m$ and is the product of the conditional likelihoods of the model and all previous observations, given each observation:

\begin{equation}
\mathcal{L}(\mathcal{M}_m|\textbf{Y}_{1:K}) = \prod_{k = 1}^{K}{\mathcal{L}(\mathcal{M}_m, \textbf{Y}_{1:k-1}|\textbf{Y}_{k})},
\end{equation}

which is equal to the product of the probabilities of each observation, given the model and the history of observations to that point:

\begin{equation}
\mathcal{L}(\mathcal{M}_m|\textbf{Y}_{1:K}) = \prod_{k = 1}^{K}{\mathbb{P}[\textbf{Y}_{k}|\mathcal{M}_m, \textbf{Y}_{1:k-1}]}
\end{equation}

Each model is defined by three functions that describe the evolution of the underlying state and the data collected on it. $y_{ijk}$ is a function of $z_{ijk}$, the true state (number or density) of $i$ in $j$ at $k$, and some species-specific detection-function parameters ${\theta_{y}}_i$:

\begin{equation}
y_{ijk} = f_y(z_{ijk}, {\theta_{y}}_i)
\end{equation}

$f_y$ is a general observation function that could take multiple forms, but must make predictions of $y_{ijk}$ as non-negative integers. The unobserved true state of species $i$ in $j$ at $k$ ($z_{ijk}$) is an evolving, non-negative variable that could be integer-valued (if the state is represented by the number of individuals) or continuous (if the state is represented by the density of individuals). In either case, we can considering a general process model $f_z$ that describes the evolution of $z_{ijk}$:

\begin{equation}
z_{ijk} = f_z({z_{ijk - \Delta t}}, \Delta t, {{\theta_z}}_i)
\end{equation}

where $\Delta t$ is the time step and ${\theta_{z}}_i$ are some species-specific process-function parameters. $f_z$ is general; it can be phenomenological or mechanistic, but it must generate appropriate values for $z_{ijk}$. Given the evolutionary aspect of the model, we must define a starting point for $z_{ijk}$. To do so, we use the general initialization function $f_{z_0}$, which only depends upon species-specific parameters ${\theta_{z_0}}_i$:

\begin{equation}
z_{ijk = 0} = f_{z_0}({\theta_{z_0}}_i)
\end{equation}


\bibliography{refs}  
\end{document}